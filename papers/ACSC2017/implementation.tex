\section{Implementation}\label{C:impl}

Our refactoring tool relies on the compiler for semantic information about programs to be refactored. We get this from two sources: the compiler can be requested (with the `-Zsave-analysis' flag) to dump information from its analysis of a program; where we needed more refactoring-specific information, we used a modified version of the compiler with callbacks to our refactoring tool from the name resolution pass.

\subsection{Renaming}

To evaluate a potential renaming, our tool starts with the save-analysis information. This allows the tool to identify the declaration of a variable (or other item) and all of its uses (in contrast with a syntactic search, the compiler can differentiate between different bindings with the same name). If the renaming is valid, then this is enough information to perform the rename. To check validity, we must re-run the compiler in order to try and resolve the name name at the declaration site. If the name does resolve, then there would be potential name conflicts. Our check prevents all super- and same-block conflicts. However, it is conservative and some valid renamings are rejected (where the existing name is not in fact used in the program after the declaration).

To guard against sub-block conflicts, we must further try to resolve the new name at every use of the variable being renamed. Only if every attempt at resolution fails can we be sure that the renaming is safe.

Whilst in theory, all these checks could be done with a single pass of the compiler, in practice Rust's name resolution is not flexible enough to check an arbitrary name, it can only check a name from the source text. Furthermore, we could at the time only observe success or failure of name resolution, not the reason why (the compiler has improved considerably since we implemented this tool). That means that to be safe, we must re-compile once for each use of the variable being renamed. This is clearly expensive.

A much better approach (but outside the scope of this project) would be to modify name resolution to allow checking for arbitrary names. This approach is taken by Gorename \cite{gorename15}.

\subsection{Inlining}

Again, inlining starts with the save-analysis data. This data allows finding the number of uses of a variable to be inlined and the mutability of its type. However, this is not enough to complete our analysis. In particular, in Rust objects can have \it{interior mutability} which is not reflected in that object's type. However, it is tracked by the compiler, so our tool can query this information. We also take account of a mutability annotation being egregious by relying on the compiler identifying such unneccessary annotations. Unfortunately, this requires running the compiler to a late stage of its analysis and thus is fairly time-consuming. Finally, we rely again on name resolution to ensure that any variables which are substituted in still resolve to their original binding.

We perform the actual inlining on the AST. Following that change, we must ensure that the fragment of the AST is properly printed back into the source text. In particular, parentheses may need to be added to ensure the correct ordering of operations due to precedence. See Figure \ref{Fig:exinline} for an example.

\begin{figure}[h]
\centering
\begin{verbatim}
Input:
fn main() {
    let a = 2 + 1;
    let _ = a * 2;
}

Output:
fn main() { let _ = (2 + 1) * 2; } // rather than 2 + 1 * 2
\end{verbatim}
\caption{Correct inlining with order of operations}
\label{Fig:exinline}
\end{figure}

\subsection{Lifetime elision and reification}

Reification of lifetime parameters was based on the implementation of error reporting for missing lifetimes in the compiler. This was somewhat complicated by the compiler's representation of lifetimes ( a combination of explicit binder structures and de Bruijn indices); converting into fresh lifetime variables again required interaction with name resolution (although note that name resolution for lifetimes is a simpler case in the Rust compiler and is handled by its own code).

In contrast, the fundamentals of elision were more complex - there is no help from the compiler here, and we only implemented for very simple cases. However, since we are only removing lifetime parameters from the source code, there is no difficulty with names.

For a problematic example, see Figure \ref{Fig:partial}. Here, 'b can be elided, but 'a cannot because if it were, the compiler would treat x and y's types as having unique lifetimes.

\begin{figure}[h]
{\verb|fn foo<'a,'b>(x: &'a Debug, y: &'a Debug, z: &'b Debug)|}\newline
becomes:\newline
{\verb|fn foo<'a>(x: &'a Debug, y: &'a Debug, z: &Debug)|}
\caption{Partial elision -- only \emph{'b} removed}
\label{Fig:partial}
\end{figure}

